\documentclass[11pt]{newlfm}
\usepackage{setspace}
\onehalfspacing
\newlfmP{leftmarginsize=.75in,leftmarginskipright=15pt,topmarginskip=2em,%
stdletter,
noheadline,nofootline,dateright,addrfromright,%
orderfromdateto,sigleft}


\nameto{Prof.~William Jorgensen}
\addrto{%\parbox{2.5in}{
Department of Chemistry\\
Yale University\\
PO Box 208107\\
New Haven, CT 06520-8107}

\namefrom{Rajarshi Guha}
\addrfrom
\phonefrom{814 404 5449}


\begin{document}
    \closeline{Yours Sincerely,}
   \greetto{Dear Prof.~Jorgensen,}
    \begin{newlfm}
      I would like to submit an article titled ``\emph{Exploring
        Uncharted Territories - Predicting Activty Cliffs in
        Structure-Activity Landscapes}'' for publication in the
      Journal of Chemical Information and Modeling.

      The article builds on previous work that defined a numerical
      quantification of structure activity landscapes, termed the
      Structure Activity Landscape Index. This (and other methods such
      as SARI) have been used to characterize SAR data using the
      landscape paradigm, allowing users to identify activity cliffs
      in their datasets. However, these approaches have primarily been
      retrospective. In this paper I highlight how SALI can be used in
      a prospective manner, such that one can predict SALI values for
      pairs of molecules, one of which is a new, untested molecule. In
      other words, the method discussed in this paper allows one to
      prospectively prioritize new molecules as likely to exhibit
      activity cliffs with one or more members of the original
      dataset. The predictive approach is based on random forest
      models and considers the data set in a pairwise fashion - thus
      making it amenable for small SAR datasets. The models exhibit
      good predictive performance, suggesting the utility of this
      approach.
%  \vskip 0.5em
    \end{newlfm}
\end{document}
%%% Local Variables: 
%%% mode: latex
%%% TeX-master: t
%%% End: 
