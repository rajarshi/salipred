\documentclass[letterpaper, 12pt]{article} 
\usepackage{url} 
\usepackage{ctable}
\usepackage{amsmath,amssymb} % for \AA etc
\usepackage[onehalfspacing]{setspace} 
\usepackage{graphicx} 
\usepackage{times}

\DeclareGraphicsExtensions{.eps,.png} \onehalfspacing

\setlength{\parindent}{0pt} 
\setlength{\parskip}{1em} 
\usepackage{anysize}

\begin{document}

I would like to thank the reviewers for their comments on the original submissions. I have prepared
a revised manuscript that address the comments and my responses are detailed below.

\fbox{\textbf{Reviewer: 1}}

\textbf{(1) Please, note that there are numerous typographical errors throughout the manuscript
  (which is quite disturbing).  \vskip 1em (2) Chosen methods and descriptors should at least be
  referenced. For example, no references are provided for BCI and CDK.}

The manuscript has been thoroughly proof read and references included where appropriate

\textbf{(3) The derivation of RF models is essentially not described (except of making reference to
  default parameters settings in R). It remains largely unclear how these models were developed. }

The text has been updated to be more specific on the development of the random forest model.

\textbf{(4) The selection of training and test sets should be clearly specified in the Methods section. For
example, how were the RF models trained and evaluated for which predictions of pair-wise SALI values
are reported in Figure 2? It is hoped that the reported SALI values were not predicted for the
training data.}

We have updated the text to note the training/test split and have include results for both sets in
the text. However, due to the similar peformance of models built using the whole dataset (and the
fact that the RF approach trains and tests on out-of-bag data implicitly), we used models built
using the entire dataset (save for some hold out molecules) in the latter half of Section 4.

\textbf{(5) No support is provided for the assumption that the RF models “are relatively robust even for
datasets of small size”. In fact, the author later on states several times that limited model
quality might be due to the use of small compound sets. Appropriate validation will be essential
here.}

We have removed this statement as it is not necessarily a function of the random forest approach,
and more a function of the nature of the dataset. Small datasets, if appropriately constructed can
lead to good RMSE's. In our case, we investigated this by sampling from the Cavalli pairwise dataset, using
different sample sizes. As shown in the attached figure, model performance is clearly a function of
sample size. Importantly, for certain small samples, the model performance can be quite good. But of
course, this is due to luck.

\textbf{(6) There is only limited statistical assessment/validation of the models. Statements made in the
text that R2 values are “reasonable” and the models perform “relatively well” are difficult to
reconcile. Rather, Table 2 indicates that the R2 values were rather poor in several instances.}

The text has been updated to read more quantitatively and avoid ambiguous statements

\textbf{(7) It appears that there are systematic prediction errors for pairs. At low value ranges (that are
not relevant for activity cliff assessment), SALI values are consistently over-predicted. By
contrast, at high value ranges (that are relevant), SALI values are consistently under-predicted, in
part significantly. This is not investigated, although the apparent errors at high value ranges are
the perhaps most critical features of the reported RF models.}

XXX

\textbf{(8) In the section “Extending a landscape”, predictions are finally reported for only three (!)
molecules taken out of each data set. Carefully put, the predictions are heterogeneous. On the basis
of the results reported for this little bit of evaluation, one would be hard pressed to make a case
for the ability of the approach to predict activity cliffs with any certainty.}

XXX

\textbf{(9) For the prediction of activity cliffs, SALI values must be calculated for compound
  pairs. This is the central idea of the approach. Accordingly, fingerprint descriptors and potency
  values must be combined in some ways.  \vskip 1em (9.1.) It is nowhere stated how activity values
  are treated for compounds forming a pair. Are activity differences calculated at the pair level?
  If so, how are activity differences compared for the calculation of pair-based SALI scores - as a
  difference of differences? It should be noted predicted potency values in Figures 4-6 are
  essentially all over the place.}

The entire premise of the proposed method is that one evaluates SALI values for a dataset that has
observed activities. We state specifically (Equation 1) how the activities are combined when
molecules are considered pairwise.

It's not clear what the reviewer means by \emph{``\ldots how are activity differences compared for
  the calculation of pair-based SALI scores \ldots''}. The SALI value itself represents a pair of
molecules and in this calculation, we consider the absolute activity difference. We do not consider
pairs of pairs

\textbf{(9.2.) To obtain a fingerprint descriptor for a pair, the author aggregates fingerprints of
  individual molecules, finally, by simple averaging}

In fact, we do not generate pairwise fingerprint descriptors. Fingerprints are only used to evaluate
the similarity term in the calculation of the SALI value.

\textbf{For comparisons of pairs, Tanimoto similarity is then calculated for averaged fingerprints,
  which presents an artificial assessment of pair similarity (one can easily come up with a few
  hypothetical compound similarity relationships that make this assessment questionable at best). As
  a consequence, activity cliffs are considered for which Tanimoto similarity calculations yield
  values between 0.2 and 0.3!}

I believe that these comments are based on the discussion in Section 3.1 and it appears that this
section of the paper was not clear.

Firstly, we have never mentioned the averaging of fingerprint descriptors. Rather, the averaging is
applied to the topological descritpors (i.e., the X variables). Specifically, we generate a
descriptor vector for a pair of molecules by averaging the (topological) descriptor vectors of the
individual members of that pair.

The goal of Section 3.1 is to indicate that even though the Y variable and the X variables encode
structural information (the former indirectly and the latter directly), there is little correlation
between the individual (aggregated, topological) descriptors and the Y variable (where sructural
information is included via fingerprints). To quantify this, we evaluate the Pearson correlation
between each (aggregated, topological) descriptor and the Y variable and show that the maximum value
is very low ($R^2 < 0.15$).

This discussion does not involve fingerprint-based Tanimoto similarities at all. In fact, if one
does indeed consider the pairwise Tanimoto values of the molecules within a dataset, we do in fact
observe an appreciable number of pairs wth $T_c > 0.8$.

In refering to pairs with $T_C$ values between 0.2 and 0.3 -- it is true that such pairs are not
really activity cliffs in the original sense of the term. As described below, defining a threshold
SALI, above which a pair represents an activity cliff, is rather subjective. 

\textbf{Thus, on the basis of these similarity calculations, one could never decide which level of
  similarity might be cliff relevant. In this respect, the descriptions of activity cliff by the
  author are quite telling – he speaks of “predicted” versus “true” activity cliffs, of “relatively
  accurate” activity cliffs, and of “most significant activity cliffs that are in fact not very
  significant activity cliffs in an absolute sense”, and so on. }

I agree that the wording of this discussion is somewhat sloppy. The text has been updated to be more
quantitative.

Yet at the same time, the concept of an activity cliff is subjective. In many cases, it is
akin to saying \emph{``I'll know it's an activity cliff when I see it''}. For example, a pair of
molecules exhibiting a $T_c = 0.9$ and 1000x difference in acivity is obviously an activity
clif. Would a pair with $T_C = 0.85$ and 50x difference in activity not be an activity cliff? From
this point of view, I do not think it unreasonable to refer somewhat qualitatively to the
``cliffness'' of an activity cliff.

Certainly, with respect to the previous comment, a pair of molecules with $T_c = 0.29$ would likely
not be regarded as a cliff - I specifically note that such a case is not a significant cliff. To
address this aspect of SALI values, I have presented plots of dissimilarity versus activity ratio
(Figures 7 and 9) which allow one to more directly determine ``significant'' versus
``insignificant'' cliffs.

\textbf{Furthermore, there might be many instances of activity and fingerprint similarity
  relationships between pairs of compounds that might yield high pair-based SALI scores dominated by
  either the activity or similarity component, without forming a “true” activity cliff. The
  conclusion of this reviewer is that it is currently not possible to capture activity cliffs with
  any degree of certainty on the basis of the pair calculations, as reported.}

While it is true that the SALI definition itself allows for high SALI values to be generated for
pairs of molecules that would be small/insignificant cliffs (highly similar structures with small
difference in activity), the methodology does let one rank pairs of molecules and as indicated in
Figures 7 and 9, one can compare the strucural similarity and activity difference (or ratio) in 2D,
allowing one to separate ``true'' cliffs from apparent ones.

\textbf{(10) It also remains largely obscure how an activity landscape should be extended following
  this approach by compounds for which no activity information is available. Should the activity
  first be predicted by an RF model and then used for pair-based SALI score calculations?  One would
  hope not.}

The original description was misleading. In fact, the approach does not allow one to obtain
predicted activity values, due to the fact that the sign of the difference in activities in the SALI
formula is ignored. While one could create another model in parallel, that predicts signed activity
differences (and thereby determine whether the activity derived from the predicted SALI value should
be less than or greater than that of the molecule with observed activity), this would probably not
be very reliable! The manuscript has been updated to remove the dicussion on extending SAR
landscapes.

\fbox{\textbf{Reviewer: 2}}

\textbf{1.  The manuscript is poorly written.  There are many typos, grammatical errors, missing
  references, and incomplete descriptions of methods and results.}

The text has been thoroughly proof read to remove typos and incorrect grammar. References have been
added and methods have been described more clearly. It's not clear which descriptions were
incomplete. However, the methodology section now reads more coherently, and some aspects are now
clearer based on other reviewer comments.

\textbf{2.  Qualitative assessments of results are not acceptable.  The authors should give
  statistical evaluations and corresponding metrics for models and their predictions.}

This was an oversight on our part. The use of qualitative terms has been removed and results and
comparisons are quantitatively characterized. For example, the random forest models built to predict
SALI values are now compared to Y-scrambled models to ensure that the the results are not purely due
to chance.

\textbf{3.  The tests sets are too small to make any meaningful conclusions.}

\textbf{4.  SALI values needs to be determined for pairs of molecules - is that not the essence of
  the methodology?}

Yes. For a given dataset, with measured activity values, one can indeed evaluate the SALI
values. The proposed methodology takes these ``measured'' SALI values for a training set and then
predicts the SALI values for a new molecule with each member of the training set. As with
traditional QSAR models, a complete validation would require one to measure the activity of the new
molecule and evaluate the actual SALI values. In absence of experimental activities for the test
set, we resort to a training/test split, which is an accepted approach in QSAR modeling.

\textbf{5.  There is no discussion/direction of how to use the methodology in new applications.}

Section 4.1 in the original manuscript specifically addresses how one might use the proposed
methodology

\end{document}
